\subsubsection{Monday - Reach milestone V1}

Actionable Items:

\begin{itemize}
    \item All that needs completion is logic to implement logic to track real time metrics. This needs to be ported over to the codebase from my prototyping code.
\end{itemize}

Success Metrics:

\begin{itemize}
    \item Code pushed to the git repository.
\end{itemize}




\subsubsection{Wednesday - Implement PREEMPT-RT kernel and quantify (Milestone V2)}

Actionable Items:

\begin{itemize}
    \item Implement the PREEMPT-RT kernel along-side the standard kernel.
    \item Quantify the performance differences with the V1 software.
\end{itemize}

Success Metrics:

\begin{itemize}
    \item Analysis logged in the project journal
\end{itemize}






\subsubsection{Friday - Snake-based segmentation \& tracking (Milestone V2)}

Actionable Items:

\begin{itemize}
    \item The V1 system uses Canny to detect edges, with the detected edges passed to OpenCV's findContours() method to extract closed loop edges. The cancellation logic then uses a minimum enclosing circle (MEC) implementation which calculates a circumcircle which completely covers the minimum area of the detected contour. The MEC implementation must be replaced with the snake method.
    \item The `least distance rule' and Kalman filter approach must be researched for the backscatter tracking implementation in the system.
    \item I will be putting the parallax part of `Project holes and mitigate parallax' task on pause until the final housing is completed and the ISA tank is functional.
\end{itemize}

Success Metrics:

\begin{itemize}
    \item Code pushed to the git repository.
\end{itemize}
