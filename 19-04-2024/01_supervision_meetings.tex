\subsubsection{09-04-2024}
We didn't have a specific meeting on Tuesday's as we usually do this week due to the ISA's morning closure. However, since I work from the ISA daily, we were able to meet up informally at multiple points throughout the week to discuss progress.

Agenda:

\begin{itemize}
    \item Multiprocessing was a massive failure!
    \begin{itemize}
        \item V1 software is complete!
        \item I realised that the current version is completely sequential, which is very inefficient since on a multicore system, such as the 4-core Raspberry Pi 4 and 5, only one core will be utilised. Python is not multithread compatible out of the box due to the Global Interpreter Lock (GIL), which is a mechanism to synchronise threads. However, the `multiprocessing' package allows for the creation of processes which are functions that is run bypassing the GIL.
        \item I have re-written the software to implement each stage (i.e., greyscale conversion, histogram equalisation, etc) of the image processing pipeline as a process which runs on different cores. Since thread safety is crucial when moving data between different threads, queues must be utilised when sending data.
        \item The performance data that I measured with the multiprocessing version was very surprising when compared to the single-core version: processing each frame took an average of 2x longer with multiprocessing, I went from an average frame processing time of 17ms to 33ms!
        \item I think the biggest reason for this slow-down is due to the queues being an inter-process communication (IPC) service, there is a lot of overhead involved with IPCs, including serialisation, deserialisation and context switching between threads.
        \item Another crucial problem with multiprocessing is the induced frame delay: capturing is a fast process that feeds into the much slower processing stage - so for example, when the 25th frame is being processed, the 100th frame is being captured, resulting in a 75-frame delay.
        \item Unfortunately, due to the much reduced performance I will not be researching multiprocessing any further. Although a lot of time had been spent on this (~3.5 days) without any meaningful performance benefit when compared to the single core program, I have learnt a lot:
        \begin{itemize}
            \item Multiprocessing processes have much more overhead to startup as an entirely new OS process is being created as each one is run.
            \item IPC services have a lot of overhead, as I mentioned before. Also, there is a lot of I/O bound tasks which can severely bottleneck multiprocessing.
            \item Pipelining and multiprocessing only works for an FPGA, where each pipeline stage is hardcoded in fabric and isn't software threads that is being scheduled to each CPU-core by an OS.
            \item (Ben mentioned this:) Never second-guess the compiler/OS when it comes to optimisations as it can optimise the code much better for runtime.
        \end{itemize}
        \item Although this was a failure, I can still write about it in my final report as I have a set of meaningful data and graphs.
    \end{itemize}

    \item ISA tank is ready for use.
    \begin{itemize}
        \item The tank is ready for use, I will be using it on Monday.
        \item While we were discussing this, Ben and I tested the prototype setup to ensure the projector and camera are both properly focused and whether the script/software all works.
    \end{itemize}
\end{itemize}

Actionable Items:

\item Install PREEMPT-RT kernel and quantify:
\begin{itemize}
    \item Since I now have the V1 software in place, I can build the real-time kernel and test with it.
\end{itemize}
