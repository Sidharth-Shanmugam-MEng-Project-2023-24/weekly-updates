\begin{table}[!h]
    \centering
    \begin{tabularx}{\textwidth}{|l|X|X|X|}
        \hline
        Date & Agenda & Actionable Items & Attendees \\
        \hline
        \hline
        Multiple points across the week with Ben and Paul separately. & 
        \begin{myitemize}
            \item Started to run into a lot of software-related issues: originally installed Ubuntu Server OS for it's lightweight nature (space-wise and processing-wise since it doesn't come with a lot of features built in, which is perfect for reduced power consumption and fast processing time). However, the kernel does not support the Raspberry Pi products of cameras. For that, I'd need Raspberry Pi OS (there is a `lite' version which is also very lightweight which is what I used), unfortunately, I had to rework a lot of the set-up steps due to various process incompatibilities (mainly networking-related and Pi Camera hardware-related) with RPi OS and Ubuntu OS.
        \end{myitemize} & 
        \begin{myitemize}
            \item Once everything is set up, I need to write a script to record bubble footage using the Pi and Pi global shutter camera from the underwater testing facility so that I can fine-tune my benchmarking script.
        \end{myitemize} &
        \begin{myitemize}
            \item Sidharth Shanmugam
            \item Paul Mitchell
            \item Benjamin Henson
        \end{myitemize} \\
        \hline
    \end{tabularx}
\end{table}

\pagebreak

\begin{table}[!h]
    \centering
    \begin{tabularx}{\textwidth}{|l|X|X|X|}
        \hline
        Continuation 1/3 \\
        \hline
        \hline
        Date & Agenda & Actionable Items & Attendees \\
        \hline
        \hline
        & 
        \begin{myitemize}
            \item I researched more into the theory of real-time Linux kernels with PREEMPT-RT, I realised that this kernel patch only implements real-time functionalities in the kernel logic to replace time-bound spinlocks with real-time mutexes. This `apparently' results in a monumental reduction in general OS latency. Normal user-space tasks such as Python scripts that the user can run are usually non-real-time, however, should be more `snappy' due to not needing to wait for kernel logic to finish before pre-emption.
        \end{myitemize} & 
        \begin{myitemize}
            \item It is possible to write kernel-based software to harness real-time features, but it is difficult, therefore I will stick with Python and userspace code, I will compare the effects of RT with the non-RT kernel with the benchmark script.
        \end{myitemize} &
        \\
        \hline
    \end{tabularx}
\end{table}

\pagebreak

\begin{table}[!h]
    \centering
    \begin{tabularx}{\textwidth}{|l|X|X|X|}
        \hline
        Continuation 2/3 \\
        \hline
        \hline
        Date & Agenda & Actionable Items & Attendees \\
        \hline
        \hline
        & 
        \begin{myitemize}
            \item I have completed a basic Python script that uses the Canny edge detection algorithm to detect backscatter from the GoPro underwater footage that Ben provided. OpenCV Contour detection is used for filling the closed loop edges from the Canny output so that I can overlay black holes on a white background to drive the DLP projector to eliminate backscatter. I have added logic to track real-time metrics such as the time it takes to process each frame, etc. The test footage isn't very good since it has a lot of artifacting, this drastically reduces the Canny's performance and increases the frame processing time.
        \end{myitemize} & 
        \begin{myitemize}
            \item I need to focus on perfecting Canny before working on the Contour detection logic. For this, I'd need better test footage, which is what I will be making the recording script for. After that I can look into other types of Contour detection algorithms, perhaps research into segmentation algorithms.
        \end{myitemize} &
        \\
        \hline
    \end{tabularx}
\end{table}

\pagebreak

\begin{table}[!h]
    \centering
    \begin{tabularx}{\textwidth}{|l|X|X|X|}
        \hline
        Continuation 3/3 \\
        \hline
        \hline
        Date & Agenda & Actionable Items & Attendees \\
        \hline
        \hline
        & 
        \begin{myitemize}
            \item I need to start work on the initial report. I have logged a lot of information in my project journal so it shouldn't be too difficult to summarise in a report.
        \end{myitemize} & 
        \begin{myitemize}
            \item I will need to create a basic plan on document structure. List out section and summarise the information that will need to go in it. Bring this basic plan to the next supervision meeting and we can refine it further. Will also need to consider the tasks and timelines, can make a simple plan of the overall tasks and we can discuss breaking this down into smaller chunks and giving each an timeline. Paul mentioned to take into account of situations when things don't go to plan by adding sufficient buffers.
        \end{myitemize} &
        \\
        \hline
    \end{tabularx}
\end{table}
