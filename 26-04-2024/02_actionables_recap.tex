\subsubsection{Recording underwater footage}

Progress Report:

\begin{itemize}
    \item \textit{Done!}
\end{itemize}

Pending Tasks:

\begin{itemize}
    \item \textit{None.}
\end{itemize}




\subsubsection{Implement PREEMPT-RT kernel and quantify}

Progress Report:

\begin{itemize}
    \item \textit{Done!}
\end{itemize}

Pending Tasks:

\begin{itemize}
    \item \textit{None.}
\end{itemize}






\subsubsection{Snake-based segmentation \& tracking}

Progress Report:

\begin{itemize}
    \item \textit{No progress due to PREEMPT-RT side-tracking.}
\end{itemize}

Pending Tasks:

\begin{itemize}
    \item The V1 system uses Canny to detect edges, with the detected edges passed to OpenCV's findContours() method to extract closed-loop edges. The cancellation logic then uses a minimum enclosing circle (MEC) implementation which calculates a circumcircle that completely covers the minimum area of the detected contour. The MEC implementation must be replaced with the snake method.
    \item The `least distance rule' and Kalman filter approach must be researched for the backscatter tracking implementation in the system.
\end{itemize}
