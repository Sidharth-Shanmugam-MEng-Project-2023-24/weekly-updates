\begin{table}[!h]
    \centering
    \begin{tabularx}{\textwidth}{|X|X|X|}
        \hline
        Actionable Item & Progress Report & Pending Tasks \\
        \hline
        \hline
        Exand research and make notes. & 
        \begin{myitemize}
            \item From the few gathered papers, I have read two and have made very in-depth notes/literature reviews in my project journal.
            \item They were bubble detection papers for extracting bubble characteristics to monitor gas seepage from the ocean floor.
            \item The papers used Canny edge detection, which is different to my starting point with Simple Blob Detection.
            \item Using Canny as a starting point, one of the papers described using the Snake-method to extract the exact outlines of the bubbles.
            \item Bubble tracking was outlined in the papers: by using a Kalman filter with the previous two captured frames, then by applying the new detection of bubbles, weighted matchings can take place using the Hungarian algorithm.
        \end{myitemize} & 
        \begin{myitemize}
            \item From these papers, it is clear that there is a basic/overall standardised method for detecting bubbles (using Canny). This is a different path compared to what I was originally pursuing with image thresholding and blob detection.
            \item I will try to experiment with the edge detection methodologies that were oulined in the reports I read - this should help me gauge the complexities involved and see the performance physically.
            \item I need to read more literature - there was one that was cited by both the papers I read, and it should go into greater depth of how Canny was employed to detect bubbles. I will be reading this next.
        \end{myitemize} \\
        \hline
        Research histograms & 
        \textit{No progress on this.} & 
        \begin{myitemize}
            \item Since the literature I'm reading focusses on Canny edge detection, there isn't any need for thresholding and blob detection, therefore I can temporarily pause the research on histograms.
            \item Once I get a clear view on how the \textit{standardised} method of bubble detection works, I can then compare it with blob detection, and when I get on to that I will resume the research on an automated thresholding method with histograms.
        \end{myitemize} \\
        \hline
    \end{tabularx}
\end{table}
